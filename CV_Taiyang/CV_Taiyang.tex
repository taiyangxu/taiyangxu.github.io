% Medium Length Graduate Curriculum Vitae
% LaTeX Template
% Version 1.2 (3/28/15)
%
% This template has been downloaded from:
% http://www.LaTeXTemplates.com
%
% Original author:
% Rensselaer Polytechnic Institute 
% (http://www.rpi.edu/dept/arc/training/latex/resumes/)
%
%
% Important note:
% This template requires the res.cls file to be in the same directory as the
% .tex file. The res.cls file provides the resume style used for structuring the
% document.
%
%%%%%%%%%%%%%%%%%%%%%%%%%%%%%%%%%%%%%%%%%%%%%%%%%%%%%%%%%%%%%%%%%%%%%%%%%%%%%%%%

%-------------------------------------------------------------------------------
%	PACKAGES AND OTHER DOCUMENT CONFIGURATIONS
%-------------------------------------------------------------------------------

%%%%%%%%%%%%%%%%%%%%%%%%%%%%%%%%%%%%%%%%%%%%%%%%%%%%%%%%%%%%%%%%%%%%%%%%%%%%%%%%
% You can have multiple style options the legal options ones are:
%
%   centered:	the name and address are centered at the top of the page 
%				(default)
%
%   line:		the name is the left with a horizontal line then the address to
%				the right
%
%   overlapped:	the section titles overlap the body text (default)
%
%   margin:		the section titles are to the left of the body text
%		
%   11pt:		use 11 point fonts instead of 10 point fonts
%
%   12pt:		use 12 point fonts instead of 10 point fonts
%
%%%%%%%%%%%%%%%%%%%%%%%%%%%%%%%%%%%%%%%%%%%%%%%%%%%%%%%%%%%%%%%%%%%%%%%%%%%%%%%%

\documentclass[margin]{res}  

% Default font is the helvetica postscript font
\usepackage{helvet}
\usepackage{xcolor}
\usepackage{framed}
\usepackage{libertine}
\usepackage{tocloft}
\usepackage{graphicx}
\usepackage{multirow}
\usepackage{tabularx}
\usepackage{enumerate}
\usepackage{datetime}
\usepackage{hyperref}\hypersetup{hidelinks}

% Increase text height
\textheight=700pt

\begin{document}

%-------------------------------------------------------------------------------
%	NAME AND ADDRESS SECTION
%-------------------------------------------------------------------------------
\name{Taiyang Xu}

% Note that addresses can be used for other contact information:
% -phone numbers
% -email addresses
% -linked-in profile

\address{{\bf Fudan University}\\220 Handan Rd\\200433, Shanghai, China}
\address{\\ \hfill homepage: \href{https://taiyangxu.github.io/}{taiyangxu.github.io} \\Email: tyxu19 (YOU KNOW) fudan (DOT) edu (DOT) cn}

% Uncomment to add a third address
%\address{Address 3 line 1\\Address 3 line 2\\Address 3 line 3}
%-------------------------------------------------------------------------------

\begin{resume}
%-------------------------------------------------------------------------------
\section{Current position}
\textbf{Fudan University}, Shanghai, China \hfill 07/2024 -- now\\
Department of Mathematics \\
Postdoctoral Fellow (Mentor: Prof. Lun Zhang)
%-------------------------------------------------------------------------------


%-------------------------------------------------------------------------------
\section{Education}
\textbf{Fudan University}, Shanghai, China \hfill 09/2019 -- 06/2024\\
Ph.D. in Mathematics, supervisor Prof. Engui Fan \\
Thesis title: ``{\sl On the long-time asymptotics of the local and nonlocal mKdV equation under the nonzero background}''
\par

\textbf{China University of Mining and Technology}, Xuzhou, China \hfill 09/2015 -- 06/2019\\ 
B.Sc. in Mathematics, Distinguished Honor \\
Thesis title: ``{\sl Inverse scattering theory and integrability on several kinds of nonlinear evolution equations}''
%-------------------------------------------------------------------------------


%-------------------------------------------------------------------------------
\section{Research interests}
Integrable PDEs, Random matrices theory, Determinantal point processes, 
Orthogonal polynomials, Asymptotic analysis, Riemann-Hilbert (RH) problems, Special functions, 
Painlev\'e equations.
%-------------------------------------------------------------------------------

%-------------------------------------------------------------------------------
\section{Research articles}
\textbf{Preprints}
\begin{enumerate}[1.]
    \item Confluent hypergeometric kernel determinant on multiple large intervals (with Lun Zhang and Zhengyang Zhao)\\
    {\sl arXiv:2508.10463}

    \item On the large-time asymptotics of the defocusing mKdV equation with step-like initial data \\
    {\sl arXiv:2204.01299}
\end{enumerate}

\textbf{Publications in refereed journals}
\begin{enumerate}[1.]
    \item Painlev\'{e} transcendents in the defocusing mKdV equation with non-zero boundary conditions (with Engui Fan and Zhaoyu Wang)\\
    {\sl Communications in Mathematical Physics}, 406 (2025), 181.

    \item Soliton resolution and asymptotic stability of $N$-soliton solutions for the defocusing mKdV equation with finite density type initial data (with Engui Fan and Zechuan Zhang)\\ 
    {\sl Physica D: Nonlinear Phenomena}, 472 (2025), 134526. 

    \item Transient asymptotics of the modified Camassa-Holm equation (with Yiling Yang and Lun Zhang) \\
    {\sl Journal of the London Mathematical Society}, 110 (2024), e12967. 

    \item On the Cauchy problem of defocusing mKdV equation with finite density initial data: long-time asymptotics in soliton-less regions (with Engui Fan and Zechuan Zhang)\\
    {\sl Journal of Differential Equations}, 372 (2023), 55--122.

    \item Large-time asymptotics to the focusing nonlocal modified Kortweg-de Vries equation with step-like boundary conditions (with Engui Fan)\\
    {\sl Studies in Applied Mathematics}, 150 (2023), 1217--1273. 
    
    \item Riemann-Hilbert approach for multisoliton solutions of generalized coupled fourth-order nonlinear Schr\"odinger equations
    (with Weiqi Peng and Shoufu Tian)\\
    {\sl Mathematical Methods in the Applied Sciences}, 43 (2020), 865--880.
\end{enumerate}
%-------------------------------------------------------------------------------

%-------------------------------------------------------------------------------
\section{Grants}
\begin{enumerate}[--]
\item Shanghai Post-doctoral Excellence Program (Grant No. 2024100) \hfill 2024 -- 2026  \\
Project: Riemann-Hilbert method for several asymptotic problems related to universality from integrable systems and random matrix theory  \\
Role: Principal Investigator

\item China Postdoctoral Science Foundation (Grant No. 2024M760480)  \hfill 2024 -- 2026\\
Project: Semiclassical asymptotics and universality for nonlinear integrable shallow water wave systems \\
Role: Principal Investigator
\end{enumerate}
%-------------------------------------------------------------------------------



%-------------------------------------------------------------------------------
\section{Teaching activities}
\textbf{2019 -- 2026 @Fudan}
\begin{enumerate}[--]
% \item Fall, 2025: TA of Advanced Calculus III (MATH130001.04).
\item Spring, 2024: TA of Methods of Asymptotic Analysis (MATH630117).
\item Fall, 2021: TA of Calculus A (MATH120021.02).
\item Spring, 2020: TA of Calculus B (MATH120004.01) (Online).
\item Fall, 2019: TA of Calculus B (MATH120003.01).
\end{enumerate}
%-------------------------------------------------------------------------------


%-------------------------------------------------------------------------------
\section{Scholarships and awards}
\textbf{2019 -- 2024 @Fudan (Doctorate)}
\begin{enumerate}[--]
    \item Graduation with Honors (Shanghai Outstanding Graduate), 2024.
    \item Scholarship provided by Huatai Securities Technology, 2023.
    \item Scholarship provided by Pacific Insurance Company, 2022. 
    \item Outstanding Doctoral Candidate Scholarship provided by Fudan University, 2021. 
    \item Doctoral Scholarship of the Year provided by Fudan University, 2019--2023. 
\end{enumerate}

\textbf{2015 -- 2019 @CUMT (Undergraduate)}
\begin{enumerate}[--]
    \item Outstanding Undergraduates in China University of Mining and Technology, 2019.
\end{enumerate}
%------------------------------------------------------------------------------


%-------------------------------------------------------------------------------
\section{Co-organized activities}
\begin{enumerate}[--]
    % \item (with Lun Zhang) RMT seminar @Fudan, 2024--2026.
    \item (with Lun Zhang) Mini-workshop on Asymptotic Analysis, Fudan University, Shanghai, China, 5th--6th \& 9th June, 2025.
\end{enumerate}
%-------------------------------------------------------------------------------

%-------------------------------------------------------------------------------
\section{Attended activities (with some talks)}
\begin{enumerate}[--]
% \item 5th ZiF Summer School ``Randomness in Physics and Mathematics: From Thermalisation in Quantum Systems to Random Matrices, Universit\"{a}t Bielefeld, Bielefeld, Germany, 25 Aug--6 Sep, 2025. \\ $\Rightarrow$ Contributed talk: ``{\sl Confluent hypergeometric kernel determinant on multiple large intervals}''
% \item Seminar @UCLouvain, Louvain-la-Neuve, Belgium, 21 August, 2025. \\ $\Rightarrow$ Invited talk: ``{\sl An introduction to Dubrovin's universality conjecture for integrable PDEs, and some progress on the Camassa-Holm equation}''
\item Universality, Nonlinearity, and Integrability, In honor of Percy Deift, Seoul, Korea, 12--16 May, 2025.
\item The 2nd Workshop on Integrable Systems and Random Matrix Theory, Dongguan, China, 5--17 Jan, 2025. \\ $\Rightarrow$ Invited talk: ``{\sl Transient asymptotics of the modified Camassa-Holm equation}''
\item Random Matrix Summer School, University of Michigan, Ann Arbor, USA, 17--28 June, 2024.
\item Random Matrices and Related Topics, Jeju island, Korea, 6--10 May, 2024.
\item The 15th Hemudu Forum on Integrable Systems, Ningbo, China, 24--26 Nov, 2023. \\  $\Rightarrow$ Contributed talk: ``{\sl Integrable PDEs with nonzero boundary conditions: large-time asymptotics}''
\item Foundations of Computational Mathematics 2023 (FoCM2023), Paris, France, 12--21 June, 2023.
\item The 13rd Hemudu Forum on Integrable Systems, Ningbo, China, 15--17 Oct, 2021.
\end{enumerate}
%-------------------------------------------------------------------------------


%-------------------------------------------------------------------------------
\section{Academic visits}
\begin{enumerate}[--]
% \item 18/08/2025 -- 24/08/2025, UCLouvain, Belgium. (Host: Christophe Charlier)
\item 31/03/2025 -- 11/04/2025, Chongqing University, China. (Host: Yiling Yang)
\end{enumerate}
(I greatly appreciate for their warm hospitality)
%-------------------------------------------------------------------------------


%-------------------------------------------------------------------------------
\section{Other presentations}
\textbf{Outreach talks}
\begin{enumerate}[--]
\item ``{\sl Some asymptotic problems in mathematical physics}'', Shanghai Institute of Technical Physics, Shanghai, China, 29th April, 2025.
\end{enumerate}

\textbf{2019 -- 2026 @Fudan Integrable Systems and Random Matrix Theory Seminar }
\begin{enumerate}[--]
\item ``{\sl Fredholm determinants from Schr\"odinger type equations, and deformation of Tracy-Widom distribution}'' (reading report), Oct, 2024.
\item ``{\sl Biorthogonal measures, polymer partition functions, and random matrices}'' (reading report), April, 2024.
\item ``{\sl Painlev\'e type asymptotics for the Camassa-Holm equation}'' (reading report), Oct, 2022.
\item ``{\sl A Riemann-Hilbert approach to Fredholm determinants of Hankel composition operators: scalar-valued kernels}'' (reading report), Sept -- Oct, 2022.
\item ``{\sl Primitive potentials and bounded solutions of the KdV equation}'' (reading report), Sept. 2022.
\item ``{\sl Soliton V. The gas: Fredholm determinants, analysis and the rapid oscillations behind the kinetic equation}'' (reading report), May -- June, 2022.
\item ``{\sl Airy kernel determinant solutions to the KdV equation and integro-differential Painlev\'e equations}'' (reading report), Mar, 2022.
\item ``{\sl The defocusing nonlinear Schr\"odinger equation with step-like oscillatory initial data}'' (reading report), Oct, 2022.
\item ``{\sl Momenta spacing distributions in anharmonic and the higher order finite temperature Airy kernel}'' (reading report), Oct, 2022.
\item ``{\sl Long-Time behavior of the non-focusing nonlinear Schr\"odinger equation -- a case study}'' (reading report), April, 2022.
\item ``{\sl On the origins of Riemann-Hilbert problems in mathematics} '' (reading report), Mar, 2022.
\end{enumerate}
%-------------------------------------------------------------------------------


%-------------------------------------------------------------------------------
\section{Status}
\textbf{China} -- citizen
%-------------------------------------------------------------------------------

%-------------------------------------------------------------------------------
\section{Languages}
\begin{enumerate}[--]
\item Chinese (native)
\item English 
\end{enumerate}
%-------------------------------------------------------------------------------

%-------------------------------------------------------------------------------
\section{Computer skills}
\LaTeX, Mathematica, Matlab, HTML, C++, Javascript
%-------------------------------------------------------------------------------
\end{resume}
\begin{flushright}
    \today
\end{flushright}
\end{document}